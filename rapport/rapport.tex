\documentclass{article}
\usepackage[utf8]{inputenc}
\usepackage[T1]{fontenc}
\usepackage[french]{babel}
\usepackage{graphicx}
\usepackage{listings}
\usepackage{color}
\usepackage{geometry}
\usepackage{array}
\geometry{hmargin=2.5cm,vmargin=3cm}
\definecolor{dkgreen}{rgb}{0,0.6,0}
\definecolor{gray}{rgb}{0.5,0.5,0.5}
\definecolor{mauve}{rgb}{0.58,0,0.82}

\lstset{frame=tb,
  language=C,
  aboveskip=3mm,
  belowskip=3mm,
  showstringspaces=false,
  columns=flexible,
  basicstyle={\small\ttfamily},
  numbers=none,
  numberstyle=\tiny\color{gray},
  keywordstyle=\color{blue},
  commentstyle=\color{dkgreen},
  stringstyle=\color{mauve},
  breaklines=true,
  breakatwhitespace=true,
  tabsize=3
}

\date{\today}
\author{Léo Check\\ Tarik Atlaoui \\ Max Eliet}

\begin{document}


\begin{titlepage}
	\enlargethispage{2cm}
	\newcommand{\HRule}{\rule{\linewidth}{0.5mm}}
	\center
	\textsc{\LARGE
	PNL - Projet 2019/2020 
	} \\[1cm]
	\HRule \\[0.4cm]
	{ \huge \bfseries le système de fichiers le plus classe du monde \\[0.15cm] }
	\HRule \\[4cm]
	\large{Léo Check \\[3mm] Tarik Atlaoui \\[3mm] Max Eliet} \\[3cm]
	05 Juin 2020 \\[3cm]
	\hfill \includegraphics[width=5cm]{logoSU.jpg}
\end{titlepage}

	\newpage
	\pagenumbering{arabic}
	\section{Etat d'avancement}
	\subsection{Liste des fonctionnalités implémentées et fonctionnelles}
	Les fonctionnaliter implémenter sont les suivante :
	\newline	
	-Rotation du système de fichiers
	\newline
	-Politique de suppression de fichiers
	\newline
	-Interaction user / fs
	\section{Structure utiliser}
	Afin de réaliser les différente fonctionaliter les structure utiliser sont les suivante :
	\begin{lstlisting}
	//pour acceder au fichier 
	struct dentry;
	//pour agire sur le fichier 
	struct inode;
	// qui a du etre exporter afin de pouvoir manipuler les fichier
	const struct inode_operations ouichefs_inode_ops;
	//afin de metre en place la politique qui continent
	struct ouichefs_politic ouichefs_politic;
	\end{lstlisting}
	\section{Rotation du système de fichiers}
	\subsection{Implémentation}
	la résolution du probléme passe par l'implémentation des 2 fonctions suivante:
	\begin{lstlisting}
		int remove_lru_file(struct dentry *root);
		int remove_LRU_file_of_dir(struct dentry *dir, int nbFiles);
	\end{lstlisting}
	Les callback des deux fonctions sont mises dans la structure \verb|ouichefs_politic| afin de pouvoir les utiliser dès qu'il y a un changement de politic par l'utilisateur.
	\newline
	Le principe des deux est le même elles parcourent tous les répertoires de façon récursif et vérifie tous les répertoires pour trouver le fichier a supprimé. 
	\newline
	La différence étant les critères de sélection du fichier
	\section{Politique de suppression de fichiers}
	\subsection{Structure utiliser}
	\subsection{Implémentation}
	La politique de suppression est LRU. Par conséquent, pour la sélection du fichier à supprimer la comparaison se fait au niveau des inodes en comparant \verb|"i_mtime.tv_sec"|
	\section{Interaction user / fs}
	\subsection{Implémentation}
	Pour les intéractions user, elles passent par la structure \verb|ouichefs_politic| , suite à un echo du nom de la fonction et du répertoire dans le fichier	"/sys/kernel/ouichefs/ouichefs" permet d'activer ou désactiver
	la politique de supprésion sur un répertoire donné.

\end{document}